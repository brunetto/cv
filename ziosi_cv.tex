\documentclass[helvetica,narrow,openbib]{europecv}
\usepackage[T1]{fontenc}
\usepackage[a4paper,top=1.27cm,left=1cm,right=1cm,bottom=2cm]{geometry}
\usepackage{ifpdf}
\usepackage{bibentry}
\usepackage[italian]{babel}
\usepackage{url}
\ifpdf
    \usepackage[pdftex]{graphicx}
\else
    \usepackage{graphicx}
\fi
\usepackage{xcolor}

\renewcommand{\ttdefault}{phv} % Uses Helvetica instead of fixed width font
\renewcommand{\emph}[1]{\textbf{#1}}
\ecvlastname{Ziosi}
\ecvfirstname{Brunetto Marco}
\ecvaddress{via Ivancich n.17, 30174, Chirignago (VE)}
\ecvtelephone{(+39) 3474958152}
%\ecvfax{(+39) 051 20 95700}
\ecvemail{brunetto.ziosi@gmail.com}
% \ecvehomepage{\url{brunettomarco.ziosi@studenti.unipd.it}}
\ecvnationality{Italian}
\ecvdateofbirth{03/05/1985} % FIXME Solo per application in Europa

% \ecvposition{PhD student since 01/2012 at Università degli Studi di Padova -
% Dipartimento di Fisica e Astronomia \textbf{PhD advisors:} Dr. Michela Mapelli (INAF-OAPd) and Prof. Giuseppe Tormen (University of Padova) 
%  \textbf{PhD Fellowship funded by:} "Strategic Research Project AACSE - Algorithms and Architectures for Computational Science and Engineering"}
% \ecvpicture[width=2.cm]{foto_cv}

\begin{document}
\selectlanguage{italian}
\ecvfootnote{Curriculum Vitae of Brunetto Marco Ziosi}

\begin{europecv}
\ecvpersonalinfo[20pt]

\ecvitem{Current Position}{PhD student since 01/2012 at Università degli Studi di Padova -
Dipartimento di Fisica e Astronomia \textbf{PhD advisors:} Dr. Michela Mapelli (INAF-OAPd) % and Prof. Giuseppe Tormen (University of Padova) % Tormen solo per l'Italia
 \textbf{PhD Fellowship funded by:} "Strategic Research Project AACSE - Algorithms and Architectures for Computational Science and Engineering"
 
 PhD Thesis submission expected in July 2015}

\ecvsection{Education \& Employments}

\ecvitem{2015/01-2015/07}{INAF-OAPd fellowship to work on the project: ``Study of gravitational 
waves sources in young star clusters by means of direct N-body simulations.''}

\ecvitem{2012-present}{PhD School in Astronomy - Dipartimento di Fisica e
Astronomia
  (Università degli Studi di Padova), Research project title: \emph{``The impact of stellar evolution and dynamics on the formation of compact-object binaries''}, supervisors: Dr. Michela Mapelli \\} %and Prof. Giuseppe Tormen\\}  % come sopra

% \ecvitem{2012}{PhD School in Astronomy - Dipartimento di Fisica e
% Astronomia
%   (Università degli Studi di Padova), Research project title: \emph{``Advanced
% algorithms for cosmological N-body simulations''} under the supervision of Prof.
% Giuseppe
%   Tormen}

\ecvitem{2007-2011}{Master Degree Thesis in Astronomy - Dipartimento di Fisica e
Astronomia
  (Università degli Studi di Padova), (108/110): `\emph{`Halo-matter
cross-correlation in
cosmological simulation''}, supervisors: Prof. Giuseppe Tormen and Prof. Ravi K. Sheth\\}


\ecvitem{2004-2007}{Bachelor Degree Thesis in Astronomy - Dipartimento di Fisica e
Astronomia (Università degli Studi di Padova), (103/110): \emph{``Studio del
profilo dei
vortici ottici con diverso momento angolare''} (Characterization of optical
vortexes with different angular momentum),
  supervisors: Prof. Cesare
Barbieri, Dr. Fabrizio Tamburini, Dr. Gabriele Anzolin\\} 

% \ecvitem{2004}{High school degree of Scientific studies\\}

% \ecvitem{Research Placements}{???
%   
% }

\ecvsection{Teaching experience}
\ecvitem{2012-2013}{
\begin{itemize}
\item Teaching assistant - Dipartimento di Fisica e Astronomia (Università degli Studi di Padova), Mathematical
\item Teaching assistant - Dipartimento di Fisica e Astronomia (Università degli Studi di Padova), Python course
\end{itemize}}


\ecvsection{Grants}
\ecvitem{2014}{CO-I of the PRIN-INAF ``Star formation and evolution in galactic nuclei.'' (PI M. Mapelli, INAF-OAPd), awarded 32k EUR for 2 years (2015-2016).}
\ecvitem{2013}{Awarded 1000 EUR to attend the Gravasco IHP trimester ``N-body gravitational 
dynamical systems From N=2 to infinity.''}

\ecvsection{Accepted computational proposals}
\ecvitem{2014}{\begin{itemize}
\item ``Young star cluster disruption by tidal fields'', 50.0k CPU hours on GPU cluster EURORA @ CINECA, PI: Mapelli, CO-I: Ziosi and Moretti
\item``Star cluster formation through merger of sub-clusters'', 50.0k CPU hours on IBM PLX cluster @ CINECA, PI: Mapelli, CO-I: Ziosi and 3 other COIs
\end{itemize}}
\ecvitem{2013}{
\begin{itemize}
\item``Investigating the statistics and parameter space of double compact object binaries in young star clusters'', 50.0k CPU hours on GPU cluster EURORA, IBM PLX cluster and  IBM Blue Gene/Q Fermi @ CINECA, PI: Ziosi, 2 COIs
\item``Making very massive stars through stellar collisions'', 50.0k CPU hours on GPU cluster EURORA and IBM PLX cluster @ CINECA, PI: Mapelli, CO-I: Ziosi and 2 other COIs
\item``Star formation in proximity of a supermassive black hole'', 2.28M CPU hours on the IBM Blue
Gene/Q Fermi cluster @ CINECA, PI: Mapelli, CO-I: Ziosi and 2 other 7 COIs
\end{itemize}}
\ecvitem{2012}{
\begin{itemize}
\item``The violent life of the Galactic Centre'', 281.6k CPU hours on IBM PLX and on the 
IBM Blue Gene/Q Fermi cluster @ CINECA, PI: Mapelli, CO-I: Ziosi and 2 other COIs
\item``Computational Frontiers of Black Hole Dynamics'', 50.0k CPU hours on IBM PLX cluster @ CINECA, PI: Ripamonti, CO-I: Ziosi and 2 other COIs
\end{itemize}}


\ecvsection{Schools and Workshops}
\ecvitem{2014}{
\begin{itemize}

\item \emph{Tools and Techniques for massive data analysis}, CINECA (Bologna), 15-16 December 2014

\item \emph{Perspectives of GPU computing in Physics and Astrophysics}, Dep. of Physics of Sapienza - Rome, 15-17 Semptember 2014 (poster)

\item \emph{Stellar N-body Dynamics}, Sexten (Italy), 8-12 September 2014 (poster)

\item \emph{Astro-GR@Rome}, Monteporzio Catone (Rome), 14-18 July 2014

\item \emph{MODEST 14 - The dance of stars: dense stellar systems from infant to old}, Bad Honnef Physics Center (Germany), 2-6 June 2014 (poster)
\end{itemize}
}

\ecvitem{2013}{
\begin{itemize}
\item \emph{Workshop on Dynamics \& Kinetic theory of self-gravitating systems}, Gravasco IHP trimester ``N-body gravitational dynamical systems From N=2 to infinity...'', Paris, 4-8 November 2013 (contributed talk)

\item\emph{Seminar on Galactic Dynamics}, Gravasco IHP trimester ``N body gravitational 
dynamical systems From N=2 to infinity...'', Paris, 21 October-1 November 2013

 \item \emph{Workshop on High Performance Scientific Computing}, Strategic Research Project AACSE, Departement of Information Engineering - Padova, 9 Semptember 2013 

 \item \emph{PhD
Summer School on High Performance Scientific Computing}, Strategic Research Project AACSE, Departement of Information Engineering - Padua, 16-18 Semptember 2013 (contributed talk)

\item \emph{INFN School Of Statistics}, Vietri sul Mare (Italy), 3-7 June 2013

  \item \emph{School on Gravitational Waves, neutrinos 
and multiwavelenght e.m. observations: the new frontier of Astronomy}, Monteporzio Catone (Rome), 10-15 April 2013

\end{itemize}
}
\ecvitem{2012}{
\begin{itemize}
\item \emph{IMPRS Summer School on Computational Astrophysics}, Heidelberg, Germany, 10-14 September
2012

 \item \emph{International School of Astrophysics on the Fundamental Cosmic
distance
scale and the Transient Sky}, Teramo, Italy, 11-15 June 2012 (contributed talk)

 \item \emph{Summer School of Parallel Computing}, CINECA (Bologna), 2-13 July 2012

 \item \emph{Introduction to C language for scientific programming},
 CINECA (Bologna), 17-18 May 2012
\end{itemize}
 }
 \ecvitem{2011}{
 \begin{itemize}
 \item \emph{PhD
Summer School on Algorithms and Architectures for Computational Science and
Engineering}, Departement of Information Engineering - Padua, 12-16 Semptember 2011  (contributed talk)

 \item \emph{Workshop on Visualization of Large scientific Data}, CINECA (Bologna), 14-15 June
2011 

 \item \emph{Python for computational science}, CINECA, 16-18 May 2011

\item \emph{Introduction to GPGPU and CUDA
programming}, CINECA (Bologna), 27 April 2011
\end{itemize}
}


\ecvsection{Other courses and experiences}
\ecvitem{2014}{Amazon Web Services Cloud School, 24 July 2014, Milano}
\ecvitem{2011}{Internship at University of Padova: ``Technical characterization of the Astrophysical Observatory of Asiago''}
\ecvitem{2009}{Certificate of photometric and spectroscopic digital analysis expert (FSE european course)}
\ecvitem{2007}{Certificate of digital data analysis expert (FSE european course)}
\ecvitem{2004}{``Il cielo come laboratorio'' (Sky as a lab), in collaboration with the Astronomy Department of the University of Padua}

\ecvsection{Publications}
\ecvitem{}{
\begin{itemize}
\item Ziosi B. M., Mapelli M., Branchesi M., Tormen G., {\it Dynamics of stellar black holes in young star clusters with different metallicities - II. Black hole-black hole binaries}, 2014, MNRAS, 441, 3703Z
\item Branchesi M. , Woan G., Astone P., Bartos I., Colla A., Covino S., Drago M., Fan X., Frasca S., Hanna C., Haskell B., Hazboun J.S., Heng I.S., Holz D.E., Johnson-McDaniel N.K., Jones I.D., Keer L., Klimenko S., Kostas G., Larson S.L., Mandel I., Mapelli M., Messenger C., Mazzolo G., Melatos A., Mohanty S., Necula V., Normandin M., Obara L., Opiela R., Owen B., Palomba C., Prodi G.A., Re V., Salemi F., Sidery T.L., Sokolowski M., Schwenzer K., Tiwari V., Tringali M.C., Vedovato G., Vousden W., Yakushin I., Zadrożny A., Ziosi B.M., {\it C7 multi-messenger astronomy of GW sources}, 2014, General Relativity and Gravitation, 46, 1771
\end{itemize}
}

\ecvsection{Scientific interests} 
\ecvitem{}{
\begin{itemize}
\item dynamics of black holes and neutron stars in star clusters 
\item direct-summation N-body simulations in star clusters 
\item gravitational waves in the frequency range of Advanced VIRGO and LIGO 
\item stellar and binary evolution
\item big data analysis and visualization
\end{itemize}}

\ecvsection{Reference persons} 
\ecvitem{}{{\bf Dr. Michela Mapelli}

INAF - Osservatorio Astronomico di Padova

Vicolo dell'Osservatorio 5

35122 PADOVA (Italy)

+39 049 8293 527

michela.mapelli@oapd.inaf.it

}
\ecvitem{}{{\bf Dr. Marica Branchesi}

University of Urbino "Carlo Bo" 

Piazza della Repubblica 13

61029 Urbino (Italy)

+39 0722 304521

marica.branchesi@uniurb.it

}
\ecvitem{}{{\bf Dr. Mario Spera}

INAF - Osservatorio Astronomico di Padova

Vicolo dell'Osservatorio 5

35122 PADOVA (Italy)

+39 049 8293 527

mario.spera@oapd.inaf.it

}
% \ecvitem{}{{\bf Dr. Fabrizio Tamburini}
% 
% TWIST OFF S.R.L.
% 
% Via della Croce Rossa 112
% 
% 35129 Padova (Italy)
% 
% +39 049 8697509
% 
% fabrizio.tamburini@gmail.com
% 
% }


\ecvsection{Language skills}
% \ecvitem{Mother tongue}{Italian}
%\ecvitem{Certificates}{English: First certificate (B2) \emph{2004}
%  - Oxford school of English }
%\ecvitem{}{ German: Zertifikat Deutsch (B1)
%  \emph{2010} - Goethe Institut}
% \ecvlanguageheader{(*)}
% \ecvlanguage{English}{B2}{C1}{B2}{B1}{B2}
% \ecvlanguage{Spanish}{A2}{B1}{B1}{A2}{A2}
% \ecvlanguagefooter[10pt]{(*)}

\ecvitem{}{Italian (Mother tongue), English (fluent)}

\ecvsection{Computer skills}
\ecvitem{OSs}{Linux (excellent), MacOS (excellent), Windows (excellent)}
\ecvitem{Scripting/Programming Languages}{Python (excellent), Go (excellent), Bash (excellent), 
C/C++, Matlab/Octave, Fortran, IDL}
\ecvitem{Data Analysis/Plotting tools}{Veusz (excellent), Matplotlib/Pylab (excellent), 
Supermongo, Gnuplot, IDL, IRAF}
\ecvitem{Experience in HPC}{
\begin{description}
	\item {\it Codes:} StarLab-GPU, Gadget
	\item {\it Libraries:} Experiences with MPI and OpenMP
	\item {\it HCP facilities I have used during my PhD:} 
		\begin{itemize}
			\item {\bf Eurora} (CINECA): Linux Cluster with 1024 cores (Intel Xeon E5-2658 and E5-2687W), 1.1 TB of RAM, 64 nVIDIA Tesla K20 (Kepler), CINECA storage, Infiniband network
			\item {\bf Green II/gStar\&swinStar} (Swinburne University): Linux cluster with 1976 cores (Intel Xeon 5650 and E5-2660), 7.904 TB of RAM, 100 NVIDIA Tesla C2070, 21NVIDIA Tesla M2090, 64 NVIDIA Tesla K10, 1.6 PB Lustre file system, QDR infiniband
			\item {\bf PLX} (CINECA): Linux cluster with 3288 cores (Intel Xeon E5645), 528 nVIDIA Tesla M2070,  20 nVIDIA Tesla M2070Q, 14 TB of RAM, CINECA storage, Infiniband network
			\item {\bf ZBOX} (University of Zurich): Linux cluster with 3072 cores (Intel Xeon E5), 12 TB of RAM, 684 TB Lustre filesystem, 800 TB tape backup, Infiniband and Gbit ethernet network
			\item {\bf monster} (Dept. of Physics and Astronomy, UniPD): Linux cluster with 104 core (Intel Xeon e5430), 464 GB of RAM, 6.39 TB storage, Gbit ethernet network
			\item {\bf SP7} (Dept. of Information Engineering, UniPD): IBM Power7 system, 92 equivalent computing cores for Linux (124 for ADA), 768 GB of RAM shared through hardware board, 1 TB storage, Infiniband network
		\end{itemize}		
\end{description}}
\ecvitem{Markup Languages/Web}{LaTeX, Markdown, HTML, Javascript}
\ecvitem{Graphics}{Inkscape, Gimp, ImageMagick, Blender}
\ecvitem{Presentation}{Beamer, Sozi/Inkscape, Prezi, PowerPoint, HTML/Javascript based tools like ImpressJS and Remark}
\ecvitem{Office/Internet}{MSOffice, LibreOffice/OpenOffice, Chrome, Internet Explorer, Firefox, Opera, Outlook, Thuderbird}
\ecvitem{Others}{Git, MySQL, PostgreSQL}

\ecvsection{Other interests}
\ecvitem{}{

Volunteering with A.G.E.S.C.I. and Protezione Civile (Civil Defense)

Digital Photograpy

Piano and guitar

Travelling

Mapping and geo-data manipulation
}

\ecvsection{Signature}
\ecvitem{}{
\begin{flushright}
\includegraphics[width=9cm]{firma}
\end{flushright}
}

\end{europecv}
\end{document} 
